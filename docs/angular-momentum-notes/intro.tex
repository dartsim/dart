\section{Introduction}
Computing angular momentum for a multibody dynamic system is very
straightforward in maximal coordinates, where angular velocity of each
rigid body is readily available. However, if you choose to use
generalized coordinates and do not want to explicitly convert velocity
to maximal coordinates at each time step, you can use the following
alternative formulation to compute angular momentum.

\section{Derivation}
We begin with the computation of angular momentum for a single
particle, $\vc{x}$. This is simply a cross product of its position
vector from the pivot ($\vc{c}$) with its velocity ($\dot{\vc{x}}$),
scaled by its mass $\mu$
\begin{equation}
\vc{L} = \mu (\vc{x} - \vc{c}) \times \dot{\vc{x}}
\end{equation}

Now let us consider the angular momentum of a multibody system with
$n$ rigid bodies connected in a hierarchical structure. We assume that
the pivot is defined at the center of mass of the system $\vc{c}$,
which could be varying over time ($\dot{\vc{c}} \neq \vc{0}$).
\begin{equation}
\vc{L} = \sum_{i=1}^n \int \!\! \int \!\! \int \mu_i (\vc{x}_i(x, y, z) - \vc{c})
\times (\dot{\vc{x}}_i(x, y, z) - \dot{\vc{c}}) dxdydz
\end{equation}
where $\vc{x}_i(x, y, z)$ denotes a particle in $i$-th rigid body. $\mu_i$ is
the infinitesimal mass of a particle in $i$-th rigid body, assuming
each rigid body has uniform density.

This equation results in four terms:
\begin{equation}
\vc{L} = \sum_{i=1}^n \int \!\! \int \!\! \int \mu_i \vc{x}_i \times
\dot{\vc{x}}_i - \sum_{i=1}^n \int \!\! \int \!\! \int \mu_i \vc{x}_i \times
\dot{\vc{c}} - \vc{c} \times \sum_{i=1}^n \int \!\! \int \!\! \int \mu_i
\dot{\vc{x}}_i + \sum_{i=1}^n \int \!\! \int \!\! \int \mu_i \vc{c}
\times \dot{\vc{c}}
\label{eqn:ang_momentum}
\end{equation}
We omit the variables of integration (\ie $dx$, $dy$, $dz$) for clarity. To
compute Equation \ref{eqn:ang_momentum} efficiently, we need to rewrite it in a form
without any integral. The first term is the most complicated one so we
will deal with it later. The second, third and fourth terms can be
simplified as follows:

\paragraph{Second Term:} The integral of $\mu_i \vc{x}_i$ over the
volume of $i$-th rigid body is equal to $m_i \vc{c}_i$, where
$m_i$ is the mass of $i$-th rigid body and $\vc{c}_i$ is its
center of mass in the world coordinates. The second term can then be
written as:
\begin{equation}
-\sum_{i=1}^n \int \!\! \int \!\! \int \mu_i \vc{x}_i \times
\dot{\vc{c}} = -\sum_{i=1}^n m_i \vc{c}_i \times \dot{\vc{c}} = -m
\vc{c} \times \dot{\vc{c}}
\end{equation}
where $m$ (without subscript $i$) is the total mass of the multibody system and $\vc{c}$ again
is its center of mass in the world coordinates.

\paragraph{Third Term:} Similarly, we can rewrite the integral part of the
third term as $m_i \dot{\vc{c}}_i$ and arrive at a simpler third term:
\begin{equation}
-\vc{c} \times \sum_{i=1}^n \int \!\! \int \!\! \int \mu_i
\dot{\vc{x}}_i = -\vc{c} \times \sum_{i=1}^n m_i \dot{\vc{c}}_i = -m
\vc{c} \times \dot{\vc{c}}
\end{equation}

\paragraph{Fourth Term:} For this term, we integrate $\mu_i$ over each rigid body and
sum up all the rigid bodies to obtain the following form:
\begin{equation}
\sum_{i=1}^n \int \!\! \int \!\! \int \mu_i \vc{c} \times
\dot{\vc{c}} = m \vc{c} \times \dot{\vc{c}}
\end{equation}

Combining these three terms, the angular momentum for the multibody
system can be simplified to 
\begin{equation}
\vc{L} = \sum_{i=1}^n \int \!\! \int \!\! \int \mu_i \vc{x}_i \times
\dot{\vc{x}}_i - m \vc{c} \times \dot{\vc{c}}
\end{equation}

The only remaining integral is in the first term. Before we work on
the math, let us first introduce a new operator ''cr()'':
\begin{equation}
\mathrm{cr}(\vc{a}\vc{b}^T) = \vc{a} \times \vc{b}
\end{equation}
where $\vc{a} \in R^{3 \times 1}$ and $\vc{b} \in R^{3 \times 1}$. Specifically,
cr() takes a 3 by 3 matrix as input and outputs a 3 by 1 vector
using the following rule:
\begin{equation}
\mathrm{cr}(A) = 
\left [
\begin{array}{c}
a_{23} - a_{32} \\
a_{31} - a_{13} \\
a_{12} - a_{21}
\end{array}
\right ],\mathrm{\;\;where\;\;} A = \left (
\begin{array}{ccc}
a_{11} & a_{12} & a_{13} \\
a_{21} & a_{22} & a_{23} \\
a_{31} & a_{32} & a_{33} 
\end{array}
\right )
\end{equation}

With this new operator, we can rewrite the first term as follows:
\begin{equation}
\sum_{i=1}^n \int \!\! \int \!\! \int \mu_i \vc{x}_i \times
\dot{\vc{x}}_i =  \sum_{i=1}^n \int \!\! \int \!\! \int \mu_i
\mathrm{cr}(\vc{x}_i \dot{\vc{x}}_i^T)
\label{eqn:first_term}
\end{equation}

We can also express $\vc{x}_i$ in terms of its coordinates in the
local frame of $i$-th rigid body: $\vc{x}_i = R_i \bar{\vc{x}}_i +
\vc{r}_i$, where $R_i$ and $r_i$ are the rotation matrix and
translation vector from the local frame of $i$-th rigid body to the
world frame. $\bar{\vc{x}}_i$ denotes the local coordinates of the
particle. Plugging this expression into Equation \ref{eqn:first_term},
we continue on our derivation of the first term:
\begin{eqnarray}
\sum_{i=1}^n \int \!\! \int \!\! \int \mu_i
\mathrm{cr}(\vc{x}_i \dot{\vc{x}}_i^T) = \sum_{i=1}^n \int \!\! \int \!\! \int \mu_i
\mathrm{cr}((R_i \bar{\vc{x}}_i + \vc{r}_i) (\bar{\vc{x}}_i^T \dot{R}_i^T +
\dot{\vc{r}}_i^T))  \nonumber \\
= \sum_{i=1}^n \int \!\! \int \!\! \int \mu_i \mathrm{cr}( R_i
\bar{\vc{x}}_i \bar{\vc{x}}_i^T \dot{R}_i^T + R_i \bar{\vc{x}}_i
\dot{\vc{r}}_i^T + \vc{r}_i \bar{\vc{x}}_i^T \dot{R}_i^T + \vc{r}_i
\dot{\vc{r}}_i^T) \nonumber \\
= \sum_{i=1}^n \mathrm{cr}(R_i \int\!\!\int\!\!\int \mu_i
\bar{\vc{x}}_i \bar{\vc{x}}_i^T \dot{R}_i^T + m_i R_i \bar{\vc{c}}_i
\dot{\vc{r}}_i^T + m_i \vc{r}_i \bar{\vc{c}}_i^T \dot{R}_i^T + m_i
\vc{r}_i \dot{\vc{r}}_i^T)
\label{eqn:first_term_derivation}
\end{eqnarray}
We denote $\bar{\vc{c}}_i$ as the center of mass of $i$-th rigid body
in its own local frame. Note that the integral of the last three terms
in Equation \ref{eqn:first_term_derivation} are now expressed in terms
of aggregated quantity $\bar{\vc{c}}_i$ and $m_i$. The only integral
term remained is the first term. We then define $M$
as the integral of outer product of $\bar{\vc{x}}_i$, which can be
precomputed based on the shape of the rigid body.
\begin{equation}
M_i = \int \!\! \int \!\! \int \bar{\vc{x}}_i \bar{\vc{x}}_i^T
\end{equation} 

Finally, we arrive at the simplified formula of angular momentum: \\
\\
\begin{equation}
\framebox{
$\vc{L} = \sum_{i=1}^n m_i \mathrm{cr}(R_i M_i \dot{R}_i^T + R_i \bar{\vc{c}}_i
\dot{\vc{r}}_i^T + \vc{r}_i \bar{\vc{c}}_i^T \dot{R}_i^T + \vc{r}_i
\dot{\vc{r}}_i^T) - m \vc{c} \times \dot{\vc{c}}$
}
\label{eqn:final}
\end{equation}


\section{Angular Momentum in Maximal Coordinates}
For the sake of completeness, we also provide a formulation
for computing angular momentum in maximal coordinates.

\begin{equation}
\vc{L} = \sum_{i=1}^n (m_i (\vc{c}_i - \vc{c}) \times (\dot{\vc{c}}_i -
\dot{\vc{c}}) + R_i \bar{I}_i R_i^T \omega_i)
\end{equation}
where $\bar{I}_i$ is the inertia matrix of $i$-th rigid body in the
local frame, which can be precomputed based on the shape of the rigid
body. $\omega_i$ is the angular momentum of $i$-th rigid body in the
world frame. We can simplify the equation to make it look similar to
Equation \ref{eqn:final}.

\begin{equation}
\framebox{
$\vc{L} = \sum_{i=1}^n (m_i \vc{c}_i \times \dot{\vc{c}}_i + R_i \bar{I}_i
R_i^T \omega_i) - m \vc{c} \times \dot{\vc{c}}$
}
\end{equation}
